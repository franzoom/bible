\section*{Proposition de plan du livre d'Ézéchiel}
\subsection*{Structure reprise du commentaire de E. Di Pede
    \footnote{\textsc{E. Di Pede}, \textit{Ézéchiel}, p.24.}
}

\leftskip=1.5cm
\begin{tabular}{|m{7cm}|m{7cm}|}
    \hline
    La gloire du Seigneur quitte le Temple (Ez~8-11)
     & La gloire du Seigneur revient dans le Temple renouvelé (Ez~43,1-12) \tabularnewline
    \hline
    Oracle contre les montagnes d'Israël (Ez~6)
     & Oracle contre la montagne de Seïr (Ez~35) et promesse de restauration des montagnes d'Israël (Ez~36,1-15) \tabularnewline
    \hline
    Jugement contre Israël (Ez~20)
     & Annonce de la restauration pour Israël (Ez~36,16-23) \tabularnewline
    \hline
    Oracle contre les rois de la dynastie de David (Ez~17 et 19)
     & Annonce d'un messie descendant de David (Ez~34,23-24; 37,24-25) \tabularnewline
    \hline
    Israël rompt l'Alliance (Ez~16; 17; 20)
     & Le Seigneur restaure l'Alliance (Ez~34,25; 37,26) \tabularnewline
    \hline
    \multicolumn{2}{|c|}{Pivot: Le rescapé de Jérusalem arrive auprès des exilés (Ez~24,26-27; 33,21-22)} \tabularnewline
    \hline
\end{tabular}

\leftskip=0cm

\vspace{2em}
\subsection*{Structure plus simple
    \footnote{Structure proposée par \textsc{Van Peursen} et citée dans \textsc{F. De Haes}, \textit{Le rouleau d'Ézéchiel}, p.29.}
}

\leftskip=1.5cm

Bloc I (Ez 1 - 24): avant la chute de Jérusalem; la malédiction de la ville.\par
\decalage | Bloc II (Ez 25-32): le jugement des nations.\par
Bloc III (Ez 33-48): après la chute de Jérusalem, la rédemption de la ville, du temple, du pays et du peuple.

\leftskip=0cm