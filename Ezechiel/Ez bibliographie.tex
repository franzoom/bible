\begin{center}\rule{0.5\linewidth}{1pt}\end{center}
\section*{Bibliographie}
\subsection*{Méthode de lecture d'un texte biblique}
\begin{itemize}[label=\texttwemoji{closed_book}]
    \item \textsc{Alter, R.}, \textit{L'art de la poésie biblique}, Lessius, «Le livre et le rouleau», 2003.
    \item \textsc{Archambault, S.}, https://evangile21.thegospelcoalition.org/article/les-methodes-litteraires-dinterpretation-de-la-bible-partie-2-lanalyse-structurelle/
    \item \textsc{Rosenberg, J.} \textit{in} \textsc{Alter, R. — Kermode, F.} (éd.), \textit{Encyclopédie littéraire de la Bible}, Bayard, 2003, p.~231-255.
\end{itemize}
\subsection*{Commentaires sur le livre d'Ézéchiel}
\begin{itemize}[label=\texttwemoji{closed_book}]
    \item \textsc{Asurmendi, J.-M.}, Le prophète Ézéchiel, Cahiers Évangile n°38, 1981.
    \item \textsc{De Haes, F.}, \textit{Le Rouleau d'Ézéchiel}, Lessius, «Le livre et le rouleau», 2019.
    \item \textsc{Di Pede, E.}, \textit{Ézéchiel}, Cerf, «Mon ABC de la Bible», 2021.
    \item \textsc{Nihan, C.}, \textit{in} \textsc{Römer, T. — Macchi, J.-D. — Nihan, C.} (éd.), \textit{Introduction à l'Ancien Testament}, Labor et Fides, 2009, p.~439-458.
\end{itemize}