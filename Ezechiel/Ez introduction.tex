\fancyfoot[C]{\textsl{ \footnotesize Ézechiel 33-37}}
\fancyfoot[R]{\footnotesize page \thepage}
%\large
\noindent
\section*{Méthode de lecture d'un texte biblique}
Dei Verbum 13 : « [..] \emph{les paroles de Dieu, passant par les langues humaines, sont devenues semblables au langage des hommes, de même que jadis le Verbe du Père éternel, ayant pris l’infirmité de notre chair, est devenu semblable aux hommes.} »\par
Donc nous avons la liberté, le droit, d’utiliser des méthodes littéraires pour lire et comprendre les textes bibliques.\\
Attention, une méthode n’est pas une fin, mais un moyen, un instrument pour comprendre le texte. Mais comme tout instrument, il faut apprendre à l’utiliser…

\subsection*{1. Le genre littéraire}
\emph{Première étape : à quoi on s’intéresse ?}\\
Quel est le genre littéraire du texte ? Histoire, texte de loi, parabole, poème, hymne, enseignement, littérature apocalyptique, prophétie, texte de sagesse ?
\subsection*{2. Analyse rédactionnelle}
\emph{Après le genre littéraire, il est important de comprendre comment le texte s’insère dans un contexte plus large}\\
\begin{itemize}[label=\ding{222}]
\item A-t-on une idée de quand le texte a été écrit, et dans quel contexte ?
\item Que se passe-t-il avant ce texte ? après ? quel est le contexte général du livre ?
\item Pouvoir justifier la cohérence générale du texte choisi
\item Les lieux et les personnages sont-ils déjà connus ? Où les trouve-t-on dans le livre, dans la Bible ?
\item Existe-t-il des textes parallèles dans le même livre, ou dans la Bible ? (en particulier la question des évangiles synoptiques)
\end{itemize}
\subsection*{3. Analyse littéraire}
Comme si on utilisait un microscope, on regarde les mots, sans nous intéresser pour le moment au sens global du texte. La précision du travail dans cette étape aidera à tirer plus tard le sens global du texte. Les crayons de couleurs sont vos amis !
\begin{itemize}[label=\ding{222}]
\item Analyse sémantique
\begin{itemize}[label=\ding{47}]
\item Colorier les mots identiques
\item Relever les répétitions et les oppositions
\item Trouver les champs lexicaux
\end{itemize}
\item Lieux, temps et personnages de la scène
\begin{itemize}[label=\ding{47}]
\item Où se déroule la scène ? Comment est appelé ce lieu ? A-t-il une signification ? Y a-t-il des changements de lieu ?
\item Quand se déroule la scène ? Ce temps a-t-il une signification particulière ?
\item Qui sont les protagonistes de la scène ? 
\end{itemize}
\end{itemize}
\subsection*{4. Analyse stylistique et rhétorique}
\emph{À présent on regarde plus en hauteur le texte pour en tirer un sens plus global.}
\begin{itemize}[label=\ding{222}]
\item Relever les éventuelles comparaisons et métaphores utilisées, voir si elles ont un sens dans le contexte biblique
\item Relever les temps des verbes
\item Relever les structures rhétoriques :
\begin{itemize}[label=\ding{47}]
\item Parallélisme:\\
\decalage Le Seigneur est bon pour Israël // \\
\decalage Le Seigneur est juste envers Juda.
\item Intensification:\\
\decalage Fais-moi connaître, Seigneur, ma fin // \\
\decalage et quelle est la mesure de mes jours. (Ps 39)
\item Structure concentrique (ou chiasme) :\\
\decalage a. Demandez, et l’on vous donnera\\
\decalage \decalage b. cherchez, et vous trouverez\\
\decalage \decalage \decalage c. frappez, et l’on ouvrira.\\
\decalage \decalage \decalage c’. car quiconque demande reçoit ;\\
\decalage \decalage b’. celui qui cherche trouve ;\\
\decalage a’. et on ouvre à celui qui frappe. (Mt 7,7-8)
\end{itemize}
\item Proposer une structure du texte
\end{itemize}
\subsection*{5. Relever les citations, et voir à quoi elles font référence}
