\documentclass[12pt,a4paper,titlepage]{article}
\author{}
\date{}

\usepackage{parskip}  %gestion des espaces entre paragraphes
\usepackage{enumitem}
\def \verseSpace{\vskip 0.45em}
\def \parSpace{\vskip 1.2em}

\def \pscolor{red} %couleur de la numérotation des versets

\usepackage{geometry}
\usepackage{xcolor}
\usepackage{graphicx}
\usepackage{multicol} 
\usepackage{microtype} %utiliser toutes les caractéristiques de la police et pour condenser les lignes
\usepackage{graphicx}
\usepackage{adjmulticol}
\usepackage{fancyhdr}

\geometry{
  twoside=false,
  left=20mm,
  right=20mm,
  top=20mm,
  bottom=20mm
  }
\usepackage{array}

%%%%% variables pour la numérotation des versets %%%%%%
\def \pslabelsep{0.2em} %distance entre le numéro de verset et le texte
\def \psleftmargin{0em} %marge à gauche lors de la numérotation des versets de psaumes



%%%% définition de la page de titre %%%%
\title{\fontsize{50}{60}\selectfont \texthebrew{יְחֶזְקֵאל}\\
\fontsize{40}{50}\selectfont \textsc{Ézéchiel} \par
\normalsize François Labadens \\
\vspace{8em}
\begin{center}
    \includegraphics[width=5cm]{../assets/seminaire_lyon.png}
\end{center} }




\begin{document}

\maketitle
%\thispagestyle{plain} % Garde le style de la première page simple


\newpage
\setcounter{page}{2} % Commence la numérotation des pages à 2 
\input{"Ez prophetisme.tex"}
\newpage
\input{"Ez chronologie.tex"}
\vspace{2em}
\input{"Ez plan"}
\newpage
\fancyfoot[C]{\textsl{ \footnotesize Ézechiel 33-37}}
\fancyfoot[R]{\footnotesize page \thepage}
%\large
\noindent
\section*{Quelques précisions de vocabulaire}

\subsection*{1. Le Nom de Dieu}
\emph{Dieu est présent par plusieurs vocables dans le texte en hébreu, prononcés différemment en fonction du contexte. La traduction en français retenue est celle encouragée par le magistère de l'Église, qui demande à ne pas prononcer littéralement le Tétragramme} (\texthebrew{יהוה})\footnote{\og{}Dans les célébrations liturgiques, dans les chants et les prières, le nom de Dieu ne doit être ni employé ni prononcé sous la forme du tétragramme YHWH.\fg{}, Congrégation pour le culte divin et la discipline des sacrements, le 29 juin 2008.}.\par
\begin{itemize}[label=\ding{222}] 
\item ~\texthebrew{יהוה} se prononce \textit{adonaï} et se traduit \textit{le Seigneur} (vient de la traduction de la Septante \textgreek{κύριος}).
\item ~\texthebrew{אֲלֵיהֶם} \raggedright se prononce \textit{elohim} et se traduit \textit{Dieu}.
\item ~\texthebrew{אֲדֹנָי} se prononce \textit{adonaï} et se traduit \textit{le Seigneur}.
\item ~\texthebrew{אֲדֹנָי יְהוִה} se prononce \textit{adonaï elohim} et se traduit \textit{le Seigneur Dieu}.
\end{itemize}

\subsection*{2. Les formules d'oracle}
\newpage
\input{"Ez_introduction.tex"}
\newpage
\input{"Ez bibliographie.tex"}
\vspace{2em}
\section*{Lecture du texte (Ézéchiel 2, puis 33 à 37)}
\textit{Le texte en hébreu est pris de la Biblia Hebraica Stuttgartensia (BHS).
La traduction française est celle de la Traduction Œcuménique de la Bible (TOB), avec quelques arrangements personnels pour mettre en valeur la structure du texte hébreu et la répétition de certains mots.}\par
\textit{La mise en page proposée veut mettre en valeur chaque unité d'une phrase.}
\newpage
\input{"Ez 33-37 TOB.tex"}

\end{document}