\documentclass[12pt,a4paper,titlepage]{article}

\usepackage{parskip}  %gestion des espaces entre paragraphes
\usepackage{enumitem}
\def \verseSpace{\vskip 0.45em}
\def \parSpace{\vskip 1.2em}
\def \decalage{\hspace*{0.5cm}~}
\def \pscolor{red} %couleur de la numérotation des versets
\usepackage{titlesec} %gestion des formats de titre
\usepackage{geometry}
\usepackage{xcolor}
\usepackage{graphicx}
%\usepackage{multicol}
\usepackage{microtype} %utiliser toutes les caractéristiques de la police et pour condenser les lignes
%\usepackage{adjmulticol}
\usepackage{fancyhdr}
\usepackage{amsmath} %pour permettre de faire des tableaux (et gérer les accolades)
\usepackage[french]{babel}
\usepackage{libertine} %police utilisée
\usepackage{pifont} %bibliothèque de symboles
\usepackage{twemojis} %bibliothèque de symboles
\usepackage[normalem]{ulem} %pour entourer les mots


\usepackage{polyglossia}
\setdefaultlanguage{french}
\setotherlanguages{hebrew,greek}
\newfontfamily\hebrewfont[Script=Hebrew]{SBL Hebrew}
\newfontfamily\greekfont[Script=Greek]{SBL Greek}

\usepackage{mdframed} %pour faire des lignes verticales
\newmdenv[ %config de l'environnement de mdframes pour avoir une barre vertical à gauche du texte
  topline=false,
  bottomline=false,
  rightline=false,
  skipabove=\topsep,
  skipbelow=\topsep,
  leftmargin=10pt,
  rightmargin=0pt,
  innertopmargin=0pt,
  innerbottommargin=0pt
]{siderules}


\geometry{
  twoside=false,
  left=10mm,
  right=10mm,
  top=15mm,
  bottom=15mm
}

\titleformat{\section}
{\Large\bfseries\scshape}{\thesection}{5em}{}


%%%gestion de en-tête et pieds de page
\pagestyle{fancy}
\fancyhf{} % Efface les en-têtes et pieds de page par défaut
\fancyfoot[R]{\footnotesize{page \thepage}} % Affiche le numéro de page centré dans le pied de page
\renewcommand{\headrulewidth}{0pt} % Supprime la ligne horizontale en haut de chaque page

%%%%% variables pour la numérotation des versets %%%%%%
\def \pslabelsep{0.2em} %distance entre le numéro de verset et le texte
\def \psleftmargin{0em} %marge à gauche lors de la numérotation des versets de psaumes
\usepackage[french]{babel}
\usepackage{libertine}
  
\title{\vspace{5cm} \huge ÉZECHIEL 33-37}
\date{Texte Français-Hébreu pour étude}

\begin{document}
\maketitle
\vfill
\thispagestyle{empty}
\newpage 

\fancyfoot[C]{\textsl{ \footnotesize Ézechiel 33-37}}
\fancyfoot[R]{\footnotesize page \thepage}
%\large
\noindent
\section*{Chapitre 33}
\begin{enumerate}[leftmargin=\psleftmargin, labelsep = \pslabelsep, label={\arabic*}, font=\color{\pscolor}\small\textsuperscript, parsep=0em, itemsep=0em, topsep=0em ]
\item Il y eut une parole du Seigneur pour moi:
\item Fils d’homme, parle aux gens de ton peuple et dis-leur: \\ Soit un pays: je fais venir contre lui l’épée.\\ Les gens de ce pays prennent parmi eux un homme et l’établissent comme guetteur.
\item Cet homme voit venir l’épée contre ce pays; il sonne du cor et avertit le peuple.
\item Quelqu’un entend bien le son du cor, mais ne tient pas compte de l’avertissement: \\ quand l’épée viendra et l’emportera, son sang sera sur sa tête. 
\item Il avait entendu le son du cor, mais n’avait pas tenu compte de l’avertissement. \\ Son sang sera sur lui. \\ Par contre celui qui aura tenu compte de l’avertissement sauvera sa vie. 
\item Mais le guetteur voit venir l’épée et ne sonne pas du cor: les gens ne sont pas avertis; \\ quand l’épée viendra et emportera l’un d’eux, \\ c’est par la faute du guetteur que cet homme sera emporté, 
\\ et je demanderai compte de son sang au guetteur. \parSpace

\item C’est donc toi, fils d’homme, que j’ai établi guetteur pour la maison d’Israël;\\ tu écouteras la parole qui sort de ma bouche et tu les avertiras de ma part.
\item Si je dis au méchant: \og{}Méchant, tu mourras certainement\fg{},\\ mais que toi, tu ne parles pas pour avertir le méchant de quitter sa conduite, \\ lui, le méchant, mourra de son péché, \\ mais c’est à toi que je demanderai compte de son sang. 
\item Par contre, si tu avertis le méchant pour qu’il se détourne de sa conduite, \\ et qu’il ne veuille pas s’en détourner, \\ il mourra de son péché, \\ et toi, tu sauveras ta vie. \parSpace

\item Écoute, fils d’homme, dis à la maison d’Israël: vous parlez ainsi: \\ \og{}Nos révoltes et nos péchés sont sur nous, \\ nous pourrissons à cause d’eux, comment pourrons-nous vivre ?\fg{} 
\item Dis-leur: Par ma vie – oracle du Seigneur Dieu –, est-ce que je prends plaisir à la mort du méchant ? \\ Bien plutôt à ce que le méchant change de conduite et qu’il vive ! \\ Revenez, revenez de votre méchante conduite: \\ pourquoi faudrait-il que vous mouriez, maison d’Israël ?
\item Toi fils d’homme, dis aux gens de ton peuple: \\ La justice du juste ne le sauvera pas le jour de sa révolte \\ et la méchanceté du méchant ne le fera pas trébucher le jour où il se détournera de sa méchanceté.\\
Le juste ne pourra pas vivre de sa justice le jour où il péchera.
\item Si je dis au juste qu’il vivra certainement \\ et que celui-ci, fort de sa justice, commette un méfait, \\ aucun de ses actes justes ne sera retenu, \\ il mourra dans le méfait qu’il aura commis.
\item Si je dis au méchant: \og{}Tu mourras certainement\fg{}, \\ et qu’il se détourne de son péché, pratique le droit et la justice,
\item s’il rend le gage, restitue ce qu’il a volé, \\ s’il marche selon les lois de la vie, en évitant de faire le mal, \\ il vivra certainement, il ne mourra pas; 
\item aucun des péchés qu’il a commis ne sera retenu contre lui; \\ il a accompli le droit et la justice; il vivra. \verseSpace
\item Les gens de ton peuple disent: \og{}La façon d’agir du Seigneur n’est pas correcte\fg{}; \\ mais n’est-ce pas leur façon d’agir à eux qui n’est pas correcte ?
\item Quand le juste se détourne de sa justice, commet un méfait et en meurt,
\item quand le méchant se détourne de sa méchanceté, pratique droit et justice et vit à cause d’eux,
\item vous dites: \og{}La façon d’agir du Seigneur n’est pas correcte !\fg{} \\ Je vous jugerai chacun selon sa conduite, maison d’Israël. \parSpace

\item La douzième année de notre déportation, le cinquième jour du dixième mois,\\ un rescapé arriva vers moi de Jérusalem pour dire:  \\ \og{}La ville est tombée !\fg{}
\item La main du Seigneur, qui avait été sur moi le soir précédant la venue du rescapé, \\ m’ouvrit la bouche au moment où il arriva vers moi, le matin. \\ Ma bouche s’ouvrit, je ne fus plus muet. \verseSpace
\item Il y eut une parole du Seigneur pour moi:
\item Fils d’homme, les habitants de ces ruines qui se trouvent sur le sol d’Israël disent: \\ “Abraham qui était seul prit possession du pays; \\ nous qui sommes nombreux, c’est à nous que le pays est donné en possession.” \verseSpace
\item C’est pourquoi, dis-leur: Ainsi parle le Seigneur Dieu: \\ Vous mangez au-dessus du sang, vous levez les yeux vers vos idoles, vous commettez des crimes \\ et vous auriez le pays en possession ! 
\item Vous vivez de l’épée; \\ vous, les femmes, vous commettez ce qui est abominable; \\ vous, les hommes, vous rendez impure la femme de votre prochain. \\ Et vous auriez le pays en possession ! \verseSpace
\item Tu leur diras ceci: Ainsi parle le Seigneur Dieu: \\ Par ma vie, ceux qui sont parmi les ruines tomberont par l’épée; \\ celui qui est dans les champs, je le donne en pâture aux bêtes sauvages; \\ ceux qui sont dans les cavernes et dans les grottes mourront de la peste.
\item Je ferai du pays une solitude désolée; l’orgueil de sa force disparaîtra; \\ les montagnes d’Israël seront désertes parce que personne n’y passera.
\item On connaîtra que je suis le Seigneur \\ quand j’aurai fait du pays une solitude désertique \\ à cause de toutes les abominations qu’ils ont commises. \verseSpace
\item Écoute, fils d’homme ! \\ Les gens de ton peuple, ceux qui bavardent sur toi le long des murs et aux portes des maisons, \\– parlant les uns avec les autres, chacun avec son frère – \\ ils disent: \og{}Venez écouter quelle parole vient de la part du Seigneur !\fg{} \verseSpace
\item Ils viendront à toi comme au rassemblement du peuple; \\ ils s’assiéront devant toi, eux, mon peuple; \\ ils écouteront tes paroles mais ne les mettront pas en pratique \\ car leur bouche est pleine des passions qu’ils veulent assouvir: \\ leur cœur suit leur profit. \verseSpace
\item Au fond, tu es pour eux comme un chant passionné, \\ d’une belle sonorité, avec un bon accompagnement. \\ Ils écoutent tes paroles mais personne ne les met en pratique. \verseSpace
\item Quand ce que tu as dit arrivera, et voilà que cela arrive,\\ ils connaîtront qu’il y avait un prophète au milieu d’eux.




\end{enumerate}



\end{document}