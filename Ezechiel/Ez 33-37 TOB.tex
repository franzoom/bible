\documentclass[12pt,a4paper,titlepage]{article}
\usepackage[french]{babel}
\usepackage{parskip}  %gestion des espaces entre paragraphes
\usepackage{enumitem}
\def \indent{\hspace*{0.8cm}[~}
\newcommand{\txtBold}[1]{\bf {#1}}
\def \verseSpace{\vskip 0.45em}
\def \parSpace{\vskip 1.2em}

%%%%% variables pour la numérotation des psaumes %%%%%%
\def \pslabelsep{0.2em} %distance entre le numéro de verset et le texte
\def \psleftmargin{0em} %marge à gauche lors de la numérotation des versets de psaumes
\def \pscolor{red} %couleur de la numérotation des versets

\usepackage{geometry}
\usepackage{xcolor}
\usepackage{graphicx}
\usepackage{multicol} 
\usepackage{microtype} %utiliser toutes les caractéristiques de la police et pour condenser les lignes
\usepackage{libertine}

\usepackage{graphicx}
\usepackage{adjmulticol}
\usepackage{fancyhdr}

%\usepackage{bidi}

\geometry{
  twoside=false,
  left=20mm,
  right=20mm,
  top=20mm,
  bottom=20mm
  }
  
\title{\vspace{5cm} \huge ÉZECHIEL 33-37}

\begin{document}
\maketitle
\vfill
\thispagestyle{empty}
\newpage 

\fancyfoot[C]{\textsl{ \footnotesize Ézechiel 33-37}}
\fancyfoot[R]{\footnotesize page \thepage}
%\large
\noindent
\section*{Chapitre 33}
\begin{enumerate}[leftmargin=\psleftmargin, labelsep = \pslabelsep, label={\arabic*}, font=\color{\pscolor}\small\textsuperscript, parsep=0em, itemsep=0em, topsep=0em ]
\item Il y eut une parole du Seigneur pour moi:  \verseSpace
\item Fils d’homme, parle aux gens de ton peuple et dis-leur: \\ Soit un pays: je fais venir contre lui l’épée.\\ Les gens de ce pays prennent parmi eux un homme et l’établissent comme guetteur. \verseSpace
\item Cet homme voit venir l’épée contre ce pays; il sonne du cor et avertit le peuple.  \verseSpace
\item Quelqu’un entend bien le son du cor, mais ne tient pas compte de l’avertissement: \\ quand l’épée viendra et l’emportera, son sang sera sur sa tête.  \verseSpace
\item Il avait entendu le son du cor, mais n’avait pas tenu compte de l’avertissement. \\ Son sang sera sur lui. \\ Par contre celui qui aura tenu compte de l’avertissement sauvera sa vie.  \verseSpace
\item Mais le guetteur voit venir l’épée et ne sonne pas du cor: les gens ne sont pas avertis; \\ quand l’épée viendra et emportera l’un d’eux, \\ c’est par la faute du guetteur que cet homme sera emporté, 
\\ et je demanderai compte de son sang au guetteur. \parSpace
\item C’est donc toi, fils d’homme, que j’ai établi guetteur pour la maison d’Israël;\\ tu écouteras la parole qui sort de ma bouche et tu les avertiras de ma part.  \verseSpace
\item Si je dis au méchant: \og{}Méchant, tu mourras certainement\fg{},\\ mais que toi, tu ne parles pas pour avertir le méchant de quitter sa conduite, \\ lui, le méchant, mourra de son péché, \\ mais c’est à toi que je demanderai compte de son sang.  \verseSpace
\item Par contre, si tu avertis le méchant pour qu’il se détourne de sa conduite, \\ et qu’il ne veuille pas s’en détourner, \\ il mourra de son péché, \\ et toi, tu sauveras ta vie. \parSpace
\item Écoute, fils d’homme, dis à la maison d’Israël: vous parlez ainsi: \\ \og{}Nos révoltes et nos péchés sont sur nous, \\ nous pourrissons à cause d’eux, comment pourrons-nous vivre ?\fg{}  \verseSpace
\item Dis-leur: Par ma vie – oracle du Seigneur Dieu –, est-ce que je prends plaisir à la mort du méchant ? \\ Bien plutôt à ce que le méchant change de conduite et qu’il vive ! \\ Revenez, revenez de votre méchante conduite: \\ pourquoi faudrait-il que vous mouriez, maison d’Israël ? \verseSpace
\item Toi fils d’homme, dis aux gens de ton peuple: \\ La justice du juste ne le sauvera pas le jour de sa révolte \\ et la méchanceté du méchant ne le fera pas trébucher le jour où il se détournera de sa méchanceté.\\
Le juste ne pourra pas vivre de sa justice le jour où il péchera. \verseSpace
\item Si je dis au juste qu’il vivra certainement \\ et que celui-ci, fort de sa justice, commette un méfait, \\ aucun de ses actes justes ne sera retenu, \\ il mourra dans le méfait qu’il aura commis. \verseSpace
\item Si je dis au méchant: \og{}Tu mourras certainement\fg{}, \\ et qu’il se détourne de son péché, pratique le droit et la justice,
\item s’il rend le gage, restitue ce qu’il a volé, \\ s’il marche selon les lois de la vie, en évitant de faire le mal, \\ il vivra certainement, il ne mourra pas; 
\item aucun des péchés qu’il a commis ne sera retenu contre lui; \\ il a accompli le droit et la justice; il vivra. \verseSpace
\item Les gens de ton peuple disent: \og{}La façon d’agir du Seigneur n’est pas correcte\fg{}; \\ mais n’est-ce pas leur façon d’agir à eux qui n’est pas correcte ? \verseSpace
\item Quand le juste se détourne de sa justice, commet un méfait et en meurt,
\item quand le méchant se détourne de sa méchanceté, pratique droit et justice et vit à cause d’eux,
\item vous dites: \og{}La façon d’agir du Seigneur n’est pas correcte !\fg{} \\ Je vous jugerai chacun selon sa conduite, maison d’Israël. \verseSpace





\end{enumerate}



\end{document}