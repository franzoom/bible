\section*{Quelques précisions de vocabulaire}

\subsection*{1. Le Nom de Dieu}
\textit{Dieu est présent par plusieurs vocables dans le texte en hébreu, prononcés différemment en fonction du contexte. La traduction en français retenue est celle encouragée par le magistère de l'Église, qui demande à ne pas prononcer littéralement le Tétragramme} (\texthebrew{יהוה})\footnote{«~Dans les célébrations liturgiques, dans les chants et les prières, le nom de Dieu ne doit être ni employé ni prononcé sous la forme du tétragramme YHWH.~», Congrégation pour le culte divin et la discipline des sacrements, le 29 juin 2008.}.\par
\begin{itemize}[label=\ding{222}] 
\item ~\texthebrew{יהוה} se prononce \textit{adonaï} et se traduit \textit{le Seigneur} (vient de la traduction de la Septante \textgreek{κύριος}).
\item ~\texthebrew{אֱלֹהִים} \raggedright se prononce \textit{elohim} et se traduit \textit{Dieu}.
\item ~\texthebrew{אֲדֹנָי} se prononce \textit{adonaï} et se traduit \textit{le Seigneur}.
\item ~\texthebrew{אֲדֹנָי יְהוִה} se prononce \textit{adonaï elohim} et se traduit \textit{le Seigneur Dieu}.
\end{itemize}

\subsection*{2. Les formules autour de la prophétie}
\begin{itemize}[label=\ding{222}] 
\item \textbf{Le prophète} se dit \texthebrew{נָבִיא}: (\textit{l'appelé}). On l'appelle aussi \texthebrew{אִישׁ הָאֱלֹהִם} (\textit{l'homme de Dieu}).
\item \textbf{La formule de la \textit{Parole-Évènement}}: \texthebrew{וַיְהִי דְבַר־יְהוָה אֵלַי לֵאמֹר} (\textit{La parole du Seigneur me fut adressée, disant…}).
\item \textbf{La formule du messager}: \texthebrew{כֹּה־אָמַר אֲדֹנָי יְהוִֹה} (\textit{Ainsi par le Seigneur Dieu}) ou \texthebrew{נְאֻם אֲדֹנָי יְהוִה} (\textit{Oracle du Seigneur Dieu}) \\
La formule du messager introduit un discours de Dieu au style direct dans la parole que le prophète adresse à son auditoire.
\item \textbf{La formule d'oracle}: \texthebrew{נְאֻם־יְהוִה} ou \texthebrew{נְאֻם אֲדֹנָי יְהוִה} (\textit{Oracle du Seigneur [Dieu]})\\
La formule d'oracle, située à l'intérieur ou à la fin du discours, certifie que c'est bien Dieu qui parle à travers son prophète. Elle permet d'appuyer le discours et d'en souligner les éléments les plus importants.
\end{itemize}

\subsection*{3. Quelques mots très utilisés}
\begin{itemize}[label=\ding{222}] 
\item ~\texthebrew{רֹעִי} (\textit{ro‘iy}): le berger
\item ~\texthebrew{צֹאן} (\textit{tso'n}): le troupeau
\item ~\texthebrew{צֹפֶה} (\textit{tsofèh}): le guetteur
\item ~\texthebrew{רָשָׁע} (\textit{rasha‘}): le méchant
\end{itemize}