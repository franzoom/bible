\fancyfoot[C]{\textsl{ \footnotesize Ézechiel 33-37}}
\fancyfoot[R]{\footnotesize page \thepage}
%\large
\noindent
\section*{Quelques précisions de vocabulaire}

\subsection*{1. Le Nom de Dieu}
\emph{Dieu est présent par plusieurs vocables dans le texte en hébreu, prononcés différemment en fonction du contexte. La traduction en français retenue est celle encouragée par le magistère de l'Église, qui demande à ne pas prononcer littéralement le Tétragramme} (\texthebrew{יהוה})\footnote{\og{}Dans les célébrations liturgiques, dans les chants et les prières, le nom de Dieu ne doit être ni employé ni prononcé sous la forme du tétragramme YHWH.\fg{}, Congrégation pour le culte divin et la discipline des sacrements, le 29 juin 2008.}.\par
\begin{itemize}[label=\ding{222}] 
\item ~\texthebrew{יהוה} se prononce \textit{adonaï} et se traduit \textit{le Seigneur} (vient de la traduction de la Septante \textgreek{κύριος}).
\item ~\texthebrew{אֲלֵיהֶם} \raggedright se prononce \textit{elohim} et se traduit \textit{Dieu}.
\item ~\texthebrew{אֲדֹנָי} se prononce \textit{adonaï} et se traduit \textit{le Seigneur}.
\item ~\texthebrew{אֲדֹנָי יְהוִה} se prononce \textit{adonaï elohim} et se traduit \textit{le Seigneur Dieu}.
\end{itemize}

\subsection*{2. Les formules d'oracle}