
\usepackage{parskip}  %gestion des espaces entre paragraphes
\usepackage{enumitem}
\def \verseSpace{\vskip 0.45em}
\def \parSpace{\vskip 1.2em}
\def \decalage{\hspace*{0.5cm}~}
\def \pscolor{red} %couleur de la numérotation des versets
\usepackage{titlesec} %gestion des formats de titre
\usepackage{geometry}
\usepackage{xcolor}
\usepackage{graphicx}
%\usepackage{multicol}
\usepackage{microtype} %utiliser toutes les caractéristiques de la police et pour condenser les lignes
%\usepackage{adjmulticol}
\usepackage{fancyhdr}
\usepackage{amsmath} %pour permettre de faire des tableaux (et gérer les accolades)
\usepackage[french]{babel}
\usepackage{libertine} %police utilisée
\usepackage{pifont} %bibliothèque de symboles
\usepackage{twemojis} %bibliothèque de symboles
\usepackage[normalem]{ulem} %pour entourer les mots


\usepackage{polyglossia}
\setdefaultlanguage{french}
\setotherlanguages{hebrew,greek}
\newfontfamily\hebrewfont[Script=Hebrew]{SBL Hebrew}
\newfontfamily\greekfont[Script=Greek]{SBL Greek}

\usepackage{mdframed} %pour faire des lignes verticales
\newmdenv[ %config de l'environnement de mdframes pour avoir une barre vertical à gauche du texte
  topline=false,
  bottomline=false,
  rightline=false,
  skipabove=\topsep,
  skipbelow=\topsep,
  leftmargin=10pt,
  rightmargin=0pt,
  innertopmargin=0pt,
  innerbottommargin=0pt
]{siderules}


\geometry{
  twoside=false,
  left=10mm,
  right=10mm,
  top=15mm,
  bottom=15mm
}

\titleformat{\section}
{\Large\bfseries\scshape}{\thesection}{5em}{}


%%%gestion de en-tête et pieds de page
\pagestyle{fancy}
\fancyhf{} % Efface les en-têtes et pieds de page par défaut
\fancyfoot[R]{\footnotesize{page \thepage}} % Affiche le numéro de page centré dans le pied de page
\renewcommand{\headrulewidth}{0pt} % Supprime la ligne horizontale en haut de chaque page